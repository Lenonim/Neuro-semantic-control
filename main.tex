% !TeX encoding = UTF-8 Unicode
% !TeX root = main.tex
% !TeX TS-program = pdflatex
%% (При смене движка необходимо удалить вспомогательные файлы *.aux *.brf *.log *.out *.synctex.gz *.toc)

\documentclass{thesisby}
\usepackage{etoolbox,ifxetex,ifluatex}
\usepackage[unicode,colorlinks=false,pagebackref]{hyperref}

%%% Проверка используемого TeX-движка %%%
\ifboolexpr{bool{xetex} or bool{luatex}}{%
  \usepackage{fontspec}
  \PassOptionsToPackage{no-math}{fontspec}     % https://tex.stackexchange.com/a/26295/104425
  \usepackage{polyglossia}%[2014/05/21]        % Поддержка многоязычности

  % fonts and languages
  \defaultfontfeatures{Ligatures=TeX,Mapping=tex-text}

  \setmainlanguage[babelshorthands = true]{russian}
  \setotherlanguage{english}

  \setmainfont{Times New Roman}
  \setmonofont{Courier New}
  \setsansfont{Arial}

  \newfontfamily\cyrillicfont[Script=Cyrillic]{Times New Roman}
  \newfontfamily\cyrillicfontsf[Script=Cyrillic]{Arial}
  \newfontfamily\cyrillicfonttt[Script=Cyrillic]{Courier New}

  \newfontfamily\englishfont{Times New Roman}
  \newfontfamily\englishfontsf{Arial}
  \newfontfamily\englishfonttt{Courier New}

  \renewcommand{\UrlFont}{\small\rmfamily\tt}
}{%
  \usepackage[T1,T2A]{fontenc}
  \usepackage[utf8]{inputenc}
  \usepackage[english, russian]{babel}
  \IfFileExists{pscyr.sty}{\usepackage{pscyr}}{}  % Подключение pscyr
}

% Для борьбы с переполнениями за счет разреженных слов в абзаце
\emergencystretch=25pt

\usepackage{enumitem}

% Счётчики.
\usepackage[figure,table]{totalcount}   % Счётчик рисунков и таблиц.
\usepackage{totcount}                   % Пакет создания счётчиков на основе последнего номера подсчитываемого элемента (может требовать дважды компилировать документ).
\usepackage{totpages}

\usepackage{microtype}

%for lists
\usepackage[ampersand]{easylist}
\ListProperties(Hide=100, Hang=false, Margin=0mm, Indent1=10.5mm, Indent2=15mm, Style*=-- ,
Style2*=$\bullet$ ,Style3*=$\circ$ ,Style4*=\tiny$\blacksquare$ )

\newenvironment{easylistNum}{
    \begin{easylist}
        \ListProperties(Hide1=0, Hang=false, Margin=0mm, Indent1=10.5mm, Indent2=15mm, Start1=1, Style*=, FinalMark={)})}
        {\ListProperties(Hide=100, Hang=false, Margin=0mm, Indent1=10.5mm, Indent2=15mm, Style*=-- , Style2*=$\bullet$ ,Style3*=$\circ$ ,Style4*=\tiny$\blacksquare$ )
    \end{easylist}}

\usepackage{amsmath, amssymb, amsfonts}
\usepackage{longtable, array}
\usepackage{graphicx, epsfig}

\usepackage{algorithm}        % Для вставки псевдокода
\usepackage{algpseudocode}    % Для вставки псевдокода

% Русская традиция начертания греческих букв
\usepackage{upgreek} % Прямые греческие ради русской традиции (в формулах записывается \alpha как \upalpha и т.д.)

\usepackage{siunitx}% For Celsium sign only

\begin{document}

\hypersetup{
pdftitle = { НЕЙРО-СЕМАНТИЧЕСКОЕ УПРАВЛЕНИЕ ДЛЯ ЗАДАЧ АСУТП},
pdfauthor = {Иванюк Дмитрий Сергеевич},
pdfsubject = {Диссертация},
pdfkeywords = {ТеХ, диссертация}
}

% !TeX encoding = UTF-8
\begin{titlepage}

\begin{center} \bfseries
 Национальная академия наук Беларуси\\
\bigskip
{ГОСУДАРСТВЕННОЕ НАУЧНОЕ УЧРЕЖДЕНИЕ}
\medskip

{<<ОБЪЕДИНЕННЫЙ ИНСТИТУТ ЭНЕРГЕТИЧЕСКИХ
И ЯДЕРНЫХ ИССЛЕДОВАНИЙ – СОСНЫ>>}
\end{center}
\medskip

\noindent На правах рукописи\\
УДК  123.456 \\
\vspace{1cm}

\begin{center}
{\large ПЕТРОВ \\ Вадим Александрович}\\ \vspace{1cm}

{\bfseries Руководство по оформлению диссертации с использованием \TeX овского класса {\itshape thesisby} версии 1.2}\\
\vspace{2cm}
Диссертация на соискание ученой степени\\
кандидата физ. - \TeX{} наук\\
\medskip

по специальности 12.34.56 \TeX ника 
\end{center}
\vspace{3cm}

\begin{tabbing}
\hspace{8cm} \= \kill \>
Научный руководитель \+ \\
д-р физ. - \TeX{} наук, профессор\\
Петров А.В.
\end{tabbing}
\vspace{5cm}

\begin{center}
 \bfseries Минск 2020
\end{center}

\end{titlepage}

\tableofcontents

\chapter*{Перечень сокращений и обозначений}
\addcontentsline{toc}{chapter}{Перечень сокращений и обозначений}

\textit{ИНС} -- искусственная нейронная сеть

\textit{НС} -- нейронная сеть

\textit{АСУТП} -- автоматизированные системы управления технологическими процессами

\textit{ПИД} -- пропорционально-интегрально-дифференциальный регулятор

\textit{SCADA} -- (Supervisory for Control And Data Acquisition) система, обеспечивающая диспетчерское управление и сбор данных, относящаяся к классу программного обеспечения для создания АСУТП

\textit{«EasyServer»} -- SCADA-система, разработанная и применяемая на ОАО «Савушкин продукт» в АСУТП

\textit{PAC} -- Programmable Automation Controller - программируемый контроллер управления технологическим процессом

\textit{PFC} -- Programmable Fieldbus Controller - программируемый контроллер управления технологическим процессом

\textit{ОС} -- операционная система

\textit{ОУ} -- объект управления

\textit{ПОУ} -- пастеризационно-охладительная установка


\chapter*{Введение}
\addcontentsline{toc}{chapter}{Введение}

В современных условиях управление производством становится все сложнее, требования к эффективности более высокими. Задачи, находящиеся на уровне АСУТП, находятся в тесном взаимодействии как с верхним уровнем (планирование производства и т.п.), так и с нижним (уровень технологического оборудования). Один из путей улучшения производства может заключаться за счет совершенствования применяемых на уровне АСУТП подходов к управлению – применению последних разработок в данной области, одной из которых является нейроуправление.

ПИ- и ПИД-регуляторы были одними из первых систем управления \cite{Omatu_Khalid_Yusof}. Они зарекомендовали себя как относительно простые и надежные системы, которые достаточно эффективно решали поставленные задачи. И в настоящее время они остаются преобладающими системами управления, несмотря на наличие в них определенных недостатков и ограничений, которых лишено нейроуправление. Новые подходы позволяют строить более эффективные системы управления по сравнению с классическими ПИД-регуляторами.

Нейроуправление – относительно молодое направление научных исследований, которое стало самостоятельным в 1988 году. Однако исследования в этой области начались гораздо раньше. Одно из определений науки «кибернетика» рассматривает ее как общую теорию управления и взаимодействия не только машин, но и биологических существ. Нейроуправление пытается реализовать данное положение через построения систем управления (систем принятия решений), которые могут обучаться во время функционирования, и таким образом, улучшать свою эффективность работы. При этом такие системы используют параллельные механизмы обработки информации, подобно мозгу живых организмов \cite{Uskov_2004}.

Долгое время была популярна идея построения совершенной системы управления – универсального контроллера, который извне выглядел бы как «черный ящик». Он мог бы использоваться для управления любыми системами, имея связи с датчиками, исполнительными механизмами, другими контроллерами и специальную связь с «модулем эффективности» – системой, которая определяет эффективность управления исходя из заданных критериев. Пользователь такой системы управления задавал бы только желаемый результат, далее обученный контроллер управлял бы самостоятельно, возможно придерживаясь сложной стратегии достижения в будущем желаемого результата. Также он бы все время корректировал свое управление исходя из реакции объекта управления для достижения максимальной эффективности. Общая схема такой системы приведена ниже.

\begin{figure}[H]
    \tikzset{
        box/.style={
                % The shape:
                rectangle,
                % The size:
                minimum size=20mm,
                % The border:
                very thick,
                draw=red!50!black!50,         % 50% red and 50% black,
                % and that mixed with 50% white
                % The filling:
                top color=white,              % a shading that is white at the top...
                bottom color=red!50!black!20, % and something else at the bottom
                % Font
                font=\itshape,
                align=center}}

    \tikzset{big_arrow/.style={-{Stealth[length=5mm, width=4mm]}}}

    \centering
    \usetikzlibrary {positioning,shapes.misc,calc,arrows.meta,arrows}
    \begin{tikzpicture}[
            right1/.style={to path={-- ++(5,0) |- (\tikztotarget)}},
            left1/.style={to path={-- ++(-5,0) |- (\tikztotarget)}}]
        \node (o1)   [box]                        {Объект\\управления};
        \node (u1)   [below=of o1,align=center]   {$\mathbf{ U(t) }$\\Эффективность};
        \node (c1)   [box,below=of u1]            {Управляющее\\устройство\\(контроллер)};

        \node (control) [right=of c1,align=center]  {$\mathbf{ u(t) }$\\Управление};
        \node (sensors) [left=of c1,align=center]   {$\mathbf{ X(t) }$\\Показания датчиков};

        \path {
            (o1)            edge[very thick]                     (u1)
            (u1)            edge[very thick, big_arrow]          (c1)
            ($ (c1.east) $) edge[very thick, big_arrow, right1]  ($ (o1.east) $)
            ($ (o1.west) $) edge[very thick, big_arrow, left1]   ($ (c1.west) $) };
    \end{tikzpicture}

    \caption{Система с подкрепляющим обучением}
    \label{fig:reinforce_learning_system}
\end{figure}

В настоящее время не только оборудование, применяемое на ОАО «Савушкин продукт», характеризуется очень высокой сложностью, но и технологические процессы также. Настройка параметров технологических линий требует наличия специалистов высокого уровня и занимает длительное время. Соответственно требования к качеству изготовления продукции очень высоки, так как от него напрямую зависит размер получаемой прибыли (выше качество – лучше потребительские характеристики товара – дольше срок хранения – более широкие возможности по географическому охвату рынка и т.д.). Нейроуправление позволяет повысить качество продукции за счет повышения эффективности управления, а также ускорить настройку параметров. Поэтому актуальной является задача применения нейроуправления для построения сложных управляющих систем на уровне АСУТП, которые были бы лишены недостатков, присущих используемым системам (на основе ПИД-регуляторов).

\chapter*{Общая характеристика работы}

\addcontentsline{toc}{chapter}{Общая характеристика работы}

\newcommand{\actuality}{\textbf{Связь работы с научными программами (проектами), темами}}
\newcommand{\aim}{\textbf{Цель, задачи, объект и предмет исследования}}
\newcommand{\novelty}{\textbf{Научная новизна}}
\newcommand{\defpositions}{\textbf{Положения, выносимые на защиту}}
\newcommand{\influence}{\textbf{Научная и практическая значимость}}
\newcommand{\reliability}{\textbf{Степень достоверности}}
\newcommand{\contribution}{\textbf{Личный вклад соискателя ученой степени в результаты диссертации}}
\newcommand{\probation}{\textbf{Апробация диссертации и информация об использовании ее результатов}}
\newcommand{\publications}{\textbf{Опубликованность результатов диссертации}}

%%http://www.linux.org.ru/forum/general/6993203#comment-6994589 (используется totcount).
\makeatletter
\def\formbytotal#1#2#3#4#5{%
    \newcount\@c
    \@c\totvalue{#1}\relax
    \newcount\@last
    \newcount\@pnul
    \@last\@c\relax
    \divide\@last 10
    \@pnul\@last\relax
    \divide\@pnul 10
    \multiply\@pnul-10
    \advance\@pnul\@last
    \multiply\@last-10
    \advance\@last\@c
    \total{#1}~#2%
    \ifnum\@pnul=1#5\else%
    \ifcase\@last#5\or#3\or#4\or#4\or#4\else#5\fi
    \fi
}
\makeatother

{\actuality}
\vspace{3mm}

Тема диссертации соответствует приоритетному направлению научно-технической деятельности согласно пункту 1 перечня приоритетных направлений научной, научно-технической и инновационной деятельности на 2021-2025 годы (Указ Президента Республики Беларусь от 07 мая 2020 г. № 156).

\vspace{3mm}
\aim
\vspace{3mm}

\textit{Целью исследования} разработка нейрорегуляторов для решения задач управления АСУТП в SCADA-системе «EasyServer», используемой на ОАО «Савушкин продукт».

Указанная цель определяет следующие \textit{задачи исследования}:
\begin{easylistNum}
    & Провести анализ существующих методов нейроуправления;
    & Провести анализ алгоритмов построения нейрорегуляторов;
    & Реализовать нейрорегулятор в виде программного модуля (язык С++) для ОС Windows;
    & Адаптировать данные модули для PLCnext;
    & Внедрить нейрорегулятор в действующие проекты;
    & Провести исследование работы нейрорегулятора как части реально действующей системы управления;
    & Провести анализ с используемыми системами управления на классических ПИД-регуляторах.
\end{easylistNum}

\textit{Объектом исследования} являются системы управления, нейрорегуляторы, ПИД-регуляторы. \textit{Предметом исследования} выступают методы и алгоритмы нейроуправления.

\vspace{3mm}
\novelty
\vspace{3mm}

Научная новизна состоит в разработке новых методов гибридного нейроуправления.

Нейроуправление – передовая область исследования в теории управления сложными динамическими системами. Применение данной технологии позволит повысить качество функционирования уже существующих систем, управлять сложными нелинейными системами с недостижимой ранее эффективностью.

\vspace{3mm}
\defpositions
\vspace{3mm}

\begin{enumerate}[wide, labelindent=10mm]
    \item Нейро-ПИД-регулятор. Он может использоваться как замена обычному ПИД-регулятору, сохраняя при этом все достоинства нейронных сетей.
    \item Исследовательский макет системы управления пастеризационной установкой. Он используется для моделирования работы системы в режиме реального времени.
\end{enumerate}

%Личный вклад
\vspace{3mm}
\contribution
\vspace{3mm}

Основные положения диссертации получены автором лично. Из публикаций, сделанных в соавторстве, в содержание диссертации включены только те результаты, которые были получены соискателем лично. Соавтором основных публикаций автора является научный руководитель д.т.н., профессор В.А. Головко, который осуществлял определение целей и постановку задач исследований, выбор методов исследований, принимал участие в планировании работ и обсуждении результатов.

\vspace{3mm}
\probation
\vspace{3mm}

Основные положения и результаты диссертационной работы докладывались и обсуждались на:
\begin{easylistNum}
    & V республиканской научной конференции молодых ученых и студентов «Современные проблемы математики и вычислительной техники» (Брест, 28-30 ноября 2007г.);
    & VII международной конференции «Развитие информатизации и государственной системы научно-технической информации РИНТИ-2009» (Минск, 16 ноября 2009 г.);
    & VI республиканской научной конференции молодых ученых и студентов «Современные проблемы математики и вычислительной техники» (Брест, 26-28 ноября 2009г.);
    & стендовой сессии "Применение нейронных сетей" на XII всероссийской научно-технической конференции «Нейроинформатика-2010» в рамках научной сессии МИФИ-2010 (Москва, 23-26 января 2010г.);
    & XIII международном симпозиуме аспирантов OWD 2011 (13th International Workshop OWD 2011, Висла, 22-25 октября 2011 г.);
    & международном конгрессе по информатике: информационные системы и технологии CSIST 2011 (Минск, 31 октября – 3 ноября 2011г.);
    & XV международном симпозиуме аспирантов OWD 2013 (15th International Workshop OWD 2013, Висла, 19-22 октября 2013 г.);
    & международной конференции «Открытые семантические технологии проектирования интеллектуальных систем» (Минск, 2017, 2018, 2019, 2020, 2021, 2022, 2023).
\end{easylistNum}

%Научная и практическая значимость
%\influence\ \ldots

%Степень достоверности
%\reliability\ полученных результатов обеспечивается \ldots \ Результаты находятся в соответствии с результатами, полученными другими авторами.

%Публикации
\vspace{3mm}
\publications
\vspace{3mm}

Результаты диссертационной работы опубликованы в 10 печатных работах. Из них 2 статьи в научном журнале объемом 0.5 авторских листа в соответствии с пунктом 18 Положения о присуждении ученых степеней и присвоении званий в Республике Беларусь, 8 статей в сборниках и материалах конференций. Без соавторов опубликовано 4 работы.
 % Характеристика работы по структуре во введении и в автореферате не отличается (ГОСТ Р 7.0.11, пункты 5.3.1 и 9.2.1), потому её загружаем из одного и того же внешнего файла, предварительно задав форму выделения некоторым параметрам.

\vspace{3mm}
\textbf{Структура и объем диссертации}
\vspace{3mm}

Диссертация состоит из введения, общей характеристики работы, четырех глав с краткими выводами по каждой главе, заключения, библиографического списка, списка публикаций автора и приложений.

В \textbf{\textit{первой главе}} рассмотрено краткое описание современных систем управления, дана классификация их типов. Определено понятие нейроуправления. Рассмотрена классификация нейросетевых приёмов управления. \textbf{\textit{Вторая глава}} посвящена рассмотрению нейросетевым моделям управления технологическими процессами. В \textbf{\textit{третьей главе}} приведены результаты разработки нейросетевой модели управления процессом пастеризации.

\textcolor{red}{
    Общий объём диссертации составляет \formbytotal{TotPages}{страниц}{у}{ы}{}, из которых \formbytotal{textpages}{страниц}{у}{ы}{} основного текста,
    \iftotalfigures
        \formbytotal{totalcount@figure}{рисун}{ок}{ка}{ков},
    \fi
    \iftotaltables
        \formbytotal{totalcount@table}{таблиц}{ей}{ами}{ами},
    \fi
    библиография из \formbytotal{citenum}{источник}{}{а}{ов}, включая \formbytotal{citenum_my} {публикац}{ия}{ии}{ий} автора.}


\chapter*{Библиографический список}
\addcontentsline{toc}{chapter}{Библиографический список}

\printbibliography[heading=subbibintoc, notkeyword={idzm}, title={Список использованных источников}]

\nocite{*}

\DeclareFieldFormat{labelnumberwidth}{#1\adddot\midsentence}
\newrefcontext[labelprefix={A-}, sorting=ynt]
\printbibliography[heading=subbibintoc, keyword={idzm}, title={Список публикаций соискателя}]


\end{document}
