\newcommand{\actuality}{\textbf{Связь работы с научными программами (проектами), темами}}
\newcommand{\aim}{\textbf{Цель, задачи, объект и предмет исследования}}
\newcommand{\novelty}{\textbf{Научная новизна}}
\newcommand{\defpositions}{\textbf{Положения, выносимые на защиту}}
\newcommand{\influence}{\textbf{Научная и практическая значимость}}
\newcommand{\reliability}{\textbf{Степень достоверности}}
\newcommand{\contribution}{\textbf{Личный вклад соискателя ученой степени в результаты диссертации}}
\newcommand{\probation}{\textbf{Апробация диссертации и информация об использовании ее результатов}}
\newcommand{\publications}{\textbf{Опубликованность результатов диссертации}}

%%http://www.linux.org.ru/forum/general/6993203#comment-6994589 (используется totcount).
\makeatletter
\def\formbytotal#1#2#3#4#5{%
    \newcount\@c
    \@c\totvalue{#1}\relax
    \newcount\@last
    \newcount\@pnul
    \@last\@c\relax
    \divide\@last 10
    \@pnul\@last\relax
    \divide\@pnul 10
    \multiply\@pnul-10
    \advance\@pnul\@last
    \multiply\@last-10
    \advance\@last\@c
    \total{#1}~#2%
    \ifnum\@pnul=1#5\else%
    \ifcase\@last#5\or#3\or#4\or#4\or#4\else#5\fi
    \fi
}
\makeatother

{\actuality}
\vspace{3mm}

Тема диссертации соответствует приоритетному направлению научно-технической деятельности согласно пункту 1 перечня приоритетных направлений научной, научно-технической и инновационной деятельности на 2021-2025 годы (Указ Президента Республики Беларусь от 07 мая 2020 г. № 156).

\vspace{3mm}
\aim
\vspace{3mm}

\textit{Целью исследования} разработка нейрорегуляторов для решения задач управления АСУТП в SCADA-системе «EasyServer», используемой на ОАО «Савушкин продукт».

Указанная цель определяет следующие \textit{задачи исследования}:
\begin{easylistNum}
    & Провести анализ существующих методов нейроуправления;
    & Провести анализ алгоритмов построения нейрорегуляторов;
    & Реализовать нейрорегулятор в виде программного модуля (язык С++) для ОС Windows;
    & Адаптировать данные модули для PLCnext;
    & Внедрить нейрорегулятор в действующие проекты;
    & Провести исследование работы нейрорегулятора как части реально действующей системы управления;
    & Провести анализ с используемыми системами управления на классических ПИД-регуляторах.
\end{easylistNum}

\textit{Объектом исследования} являются системы управления, нейрорегуляторы, ПИД-регуляторы. \textit{Предметом исследования} выступают методы и алгоритмы нейроуправления.

\vspace{3mm}
\novelty
\vspace{3mm}

Научная новизна состоит в разработке новых методов гибридного нейроуправления.

Нейроуправление – передовая область исследования в теории управления сложными динамическими системами. Применение данной технологии позволит повысить качество функционирования уже существующих систем, управлять сложными нелинейными системами с недостижимой ранее эффективностью.

\vspace{3mm}
\defpositions
\vspace{3mm}

\begin{enumerate}[wide, labelindent=10mm]
    \item Нейро-ПИД-регулятор. Он может использоваться как замена обычному ПИД-регулятору, сохраняя при этом все достоинства нейронных сетей.
    \item Исследовательский макет системы управления пастеризационной установкой. Он используется для моделирования работы системы в режиме реального времени.
\end{enumerate}

%Личный вклад
\vspace{3mm}
\contribution
\vspace{3mm}

Основные положения диссертации получены автором лично. Из публикаций, сделанных в соавторстве, в содержание диссертации включены только те результаты, которые были получены соискателем лично. Соавтором основных публикаций автора является научный руководитель д.т.н., профессор В.А. Головко, который осуществлял определение целей и постановку задач исследований, выбор методов исследований, принимал участие в планировании работ и обсуждении результатов.

\vspace{3mm}
\probation
\vspace{3mm}

Основные положения и результаты диссертационной работы докладывались и обсуждались на:
\begin{easylistNum}
    & V республиканской научной конференции молодых ученых и студентов «Современные проблемы математики и вычислительной техники» (Брест, 28-30 ноября 2007г.);
    & VII международной конференции «Развитие информатизации и государственной системы научно-технической информации РИНТИ-2009» (Минск, 16 ноября 2009 г.);
    & VI республиканской научной конференции молодых ученых и студентов «Современные проблемы математики и вычислительной техники» (Брест, 26-28 ноября 2009г.);
    & стендовой сессии "Применение нейронных сетей" на XII всероссийской научно-технической конференции «Нейроинформатика-2010» в рамках научной сессии МИФИ-2010 (Москва, 23-26 января 2010г.);
    & XIII международном симпозиуме аспирантов OWD 2011 (13th International Workshop OWD 2011, Висла, 22-25 октября 2011 г.);
    & международном конгрессе по информатике: информационные системы и технологии CSIST 2011 (Минск, 31 октября – 3 ноября 2011г.);
    & XV международном симпозиуме аспирантов OWD 2013 (15th International Workshop OWD 2013, Висла, 19-22 октября 2013 г.);
    & международной конференции «Открытые семантические технологии проектирования интеллектуальных систем» (Минск, 2017, 2018, 2019, 2020, 2021, 2022, 2023).
\end{easylistNum}

%Научная и практическая значимость
%\influence\ \ldots

%Степень достоверности
%\reliability\ полученных результатов обеспечивается \ldots \ Результаты находятся в соответствии с результатами, полученными другими авторами.

%Публикации
\vspace{3mm}
\publications
\vspace{3mm}

Результаты диссертационной работы опубликованы в 10 печатных работах. Из них 2 статьи в научном журнале объемом 0.5 авторских листа в соответствии с пунктом 18 Положения о присуждении ученых степеней и присвоении званий в Республике Беларусь, 8 статей в сборниках и материалах конференций. Без соавторов опубликовано 4 работы.
 % Характеристика работы по структуре во введении и в автореферате не отличается (ГОСТ Р 7.0.11, пункты 5.3.1 и 9.2.1), потому её загружаем из одного и того же внешнего файла, предварительно задав форму выделения некоторым параметрам.

\vspace{3mm}
\textbf{Структура и объем диссертации}
\vspace{3mm}

Диссертация состоит из введения, общей характеристики работы, четырех глав с краткими выводами по каждой главе, заключения, библиографического списка, списка публикаций автора и приложений.

В \textbf{\textit{первой главе}} рассмотрено краткое описание современных систем управления, дана классификация их типов. Определено понятие нейроуправления. Рассмотрена классификация нейросетевых приёмов управления. \textbf{\textit{Вторая глава}} посвящена рассмотрению нейросетевым моделям управления технологическими процессами. В \textbf{\textit{третьей главе}} приведены результаты разработки нейросетевой модели управления процессом пастеризации.

\textcolor{red}{
    Общий объём диссертации составляет \formbytotal{TotPages}{страниц}{у}{ы}{}, из которых \formbytotal{textpages}{страниц}{у}{ы}{} основного текста,
    \iftotalfigures
        \formbytotal{totalcount@figure}{рисун}{ок}{ка}{ков},
    \fi
    \iftotaltables
        \formbytotal{totalcount@table}{таблиц}{а}{ы}{},
    \fi
    библиография из \formbytotal{citenum}{источник}{}{а}{ов}, включая \formbytotal{citenum_my}{публикац}{ия}{ии}{ий} автора.}
