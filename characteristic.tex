{\actuality}
\vspace{3mm}

Тема диссертации соответствует приоритетному направлению научно-технической деятельности согласно пункту 1 перечня приоритетных направлений научной, научно-технической и инновационной деятельности на 2021-2025 годы (Указ Президента Республики Беларусь от 07 мая 2020 г. № 156).

\vspace{3mm}
\aim
\vspace{3mm}

\textit{Целью исследования} разработка нейрорегуляторов для решения задач управления АСУТП в SCADA-системе «EasyServer», используемой на ОАО «Савушкин продукт».

Указанная цель определяет следующие \textit{задачи исследования}:
\begin{easylistNum}
    & Провести анализ существующих методов нейроуправления;
    & Провести анализ алгоритмов построения нейрорегуляторов;
    & Реализовать нейрорегулятор в виде программного модуля (язык С++) для ОС Windows;
    & Адаптировать данные модули для PLCnext;
    & Внедрить нейрорегулятор в действующие проекты;
    & Провести исследование работы нейрорегулятора как части реально действующей системы управления;
    & Провести анализ с используемыми системами управления на классических ПИД-регуляторах.
\end{easylistNum}

\textit{Объектом исследования} являются системы управления, нейрорегуляторы, ПИД-регуляторы. \textit{Предметом исследования} выступают методы и алгоритмы нейроуправления.

\vspace{3mm}
\novelty
\vspace{3mm}

Научная новизна состоит в разработке новых методов гибридного нейроуправления.

Нейроуправление – передовая область исследования в теории управления сложными динамическими системами. Применение данной технологии позволит повысить качество функционирования уже существующих систем, управлять сложными нелинейными системами с недостижимой ранее эффективностью.

\vspace{3mm}
\defpositions
\vspace{3mm}

\begin{enumerate}[wide, labelindent=10mm]
    \item Нейро-ПИД-регулятор. Он может использоваться как замена обычному ПИД-регулятору, сохраняя при этом все достоинства нейронных сетей.
    \item Исследовательский макет системы управления пастеризационной установкой. Он используется для моделирования работы системы в режиме реального времени.
\end{enumerate}

%Личный вклад
\vspace{3mm}
\contribution
\vspace{3mm}

Основные положения диссертации получены автором лично. Из публикаций, сделанных в соавторстве, в содержание диссертации включены только те результаты, которые были получены соискателем лично. Соавтором основных публикаций автора является научный руководитель д.т.н., профессор В.А. Головко, который осуществлял определение целей и постановку задач исследований, выбор методов исследований, принимал участие в планировании работ и обсуждении результатов.

\vspace{3mm}
\probation
\vspace{3mm}

Основные положения и результаты диссертационной работы докладывались и обсуждались на:
\begin{easylistNum}
    & V республиканской научной конференции молодых ученых и студентов «Современные проблемы математики и вычислительной техники» (Брест, 28-30 ноября 2007г.);
    & VII международной конференции «Развитие информатизации и государственной системы научно-технической информации РИНТИ-2009» (Минск, 16 ноября 2009 г.);
    & VI республиканской научной конференции молодых ученых и студентов «Современные проблемы математики и вычислительной техники» (Брест, 26-28 ноября 2009г.);
    & стендовой сессии "Применение нейронных сетей" на XII всероссийской научно-технической конференции «Нейроинформатика-2010» в рамках научной сессии МИФИ-2010 (Москва, 23-26 января 2010г.);
    & XIII международном симпозиуме аспирантов OWD 2011 (13th International Workshop OWD 2011, Висла, 22-25 октября 2011 г.);
    & международном конгрессе по информатике: информационные системы и технологии CSIST 2011 (Минск, 31 октября – 3 ноября 2011г.);
    & XV международном симпозиуме аспирантов OWD 2013 (15th International Workshop OWD 2013, Висла, 19-22 октября 2013 г.);
    & международной конференции «Открытые семантические технологии проектирования интеллектуальных систем» (Минск, 2017, 2018, 2019, 2020, 2021, 2022, 2023).
\end{easylistNum}

%Научная и практическая значимость
%\influence\ \ldots

%Степень достоверности
%\reliability\ полученных результатов обеспечивается \ldots \ Результаты находятся в соответствии с результатами, полученными другими авторами.

%Публикации
\vspace{3mm}
\publications
\vspace{3mm}

Результаты диссертационной работы опубликованы в 10 печатных работах. Из них 2 статьи в научном журнале объемом 0.5 авторских листа в соответствии с пунктом 18 Положения о присуждении ученых степеней и присвоении званий в Республике Беларусь, 8 статей в сборниках и материалах конференций. Без соавторов опубликовано 4 работы.
